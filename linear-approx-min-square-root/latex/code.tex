\section{Код численного метода}

\subsection{Реализация и анализ метода наименьших квадратов}
В этом разделе представлена функция для аппроксимации данных методом наименьших квадратов, написанная на языке программирования C++. Функция принимает на вход два вектора (динамические массивы) вещественных чисел, представляющих собой наборы данных x и y, возвращает максимальное квадратичное отклонение, что помогает оценить наихудший случай ошибки аппроксимации.

\subsection{Код функции}
\begin{lstlisting}
double approximate_linear_least_squares(
    const vector<double>& xVals, 
    const vector<double>& yVals
) {
    int n = xVals.size();    

    // Check if there are enough points for approximation
    if (n < 2) return 0.0;
    
    // Calculating necessary sums for coefficients
    double xSum = accumulate(xVals.begin(), xVals.end(), 0.0);
    double ySum = accumulate(yVals.begin(), yVals.end(), 0.0);
    double xySum = inner_product(xVals.begin(), xVals.end(), yVals.begin(), 0.0);
    double xSqSum = inner_product(xVals.begin(), xVals.end(), xVals.begin(), 0.0);
    
    // Calculating the denominator
    double denominator = n * xSqSum - xSum * xSum;
    
    // Check if denominator is too close to zero (vertical line)
    if (abs(denominator) < 1e-10) return 0.0;
    
    // Calculating coefficients a and b for the linear equation
    double aCoeff = (n * xySum - xSum * ySum) / denominator;
    double bCoeff = (ySum * xSqSum - xSum * xySum) / denominator;
    
    // Finding the maximum squared deviation
    double maxSqDev = 0.0;
    for (size_t i = 0; i < n; ++i) {
        double approximateY = aCoeff * xVals[i] + bCoeff;
        double sqDev = pow(yVals[i] - approximateY, 2);
        maxSqDev = max(maxSqDev, sqDev);
    }
    
    return maxSqDev;
}
\end{lstlisting}
