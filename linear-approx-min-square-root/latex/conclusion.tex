\section{Вывод}

В данной работе был реализован метод наименьших квадратов для линейной аппроксимации набора данных. Метод показал хорошие результаты на различных наборах данных, включая линейные и нелинейные зависимости.

При анализе результатов было обнаружено, что метод дает точную аппроксимацию для линейных зависимостей (пример 1), но может давать значительные отклонения для нелинейных зависимостей (пример 2). На случайных данных метод показывает приемлемую точность (пример 3).

Метод корректно обрабатывает граничные случаи, такие как недостаточное количество точек для аппроксимации (пример 4) и вертикальные прямые (пример 5).

В сравнении с другими методами аппроксимации, метод наименьших квадратов является простым и эффективным для линейных зависимостей, но может уступать более сложным методам при аппроксимации нелинейных зависимостей.

Метод наименьших квадратов применим для задач линейной аппроксимации, где требуется найти наилучшую линейную функцию, описывающую набор данных. Он широко используется в статистике, машинном обучении и анализе данных.

Алгоритмическая сложность метода составляет O(n), где n - количество точек данных. Это означает, что время выполнения алгоритма линейно зависит от размера входных данных.

Численная ошибка метода зависит от точности вычислений и может быть уменьшена за счет использования более точных типов данных и алгоритмов вычисления сумм.

В целом, метод наименьших квадратов является надежным и эффективным инструментом для линейной аппроксимации данных и может быть успешно применен в различных областях.
