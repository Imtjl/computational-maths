\section{Описание метода наименьших квадратов}

\subsection{Постановка задачи}
Линейная аппроксимация — это процесс приближения заданной функции или имеющегося набора данных \emph{линейной} функцией $y = ax + b$. Задача состоит в том, чтобы найти коэффициенты $a$ и $b$, где роль метода наименьших квадратов (МНК) - в подборе этих коэффициентов таким образом, чтобы они минимизировали сумму квадратов разностей между наблюдаемыми значениями $y_i$ и значениями, предсказанными получившейся линейной функцией, на основе соответствующих значений $x_i$.

\subsection{Вывод формул для коэффициентов $a$ и $b$}
\subsubsection{Обозначения и предварительные вычисления}
Пусть у нас есть набор данных $\{(x_i, y_i)\}_{i=1}^n$. Требуется найти коэффициенты $a$ и $b$ для функции $y = ax + b$, которые минимизируют функцию двух переменных $F(a, b) = \sum_{i=1}^{n}(y_i - (ax_i + b))^2$.
\vspace{1mm}

Определим следующие вспомогательные величины: \\
- Сумма $x$-ов: $S_x = \sum_{i=1}^{n} x_i$, \\
- Сумма $y$-ов: $S_y = \sum_{i=1}^{n} y_i$, \\
- Сумма произведений $x$-ов и $y$-ов: $S_{xy} = \sum_{i=1}^{n} x_i y_i$, \\
- Сумма квадратов $x$-ов: $S_{xx} = \sum_{i=1}^{n} x_i^2$.

\subsubsection{Вывод линейных уравнений}
Для минимизации $F(a, b)$, находим частные производные по $a$ и $b$ и приравниваем их к нулю.

Частная производная по $a$:
\[
\frac{\partial F}{\partial a} = -2 \sum_{i=1}^{n} x_i(y_i - ax_i - b) = -2(S_{xy} - aS_{xx} - bS_x) = 0.
\]

Частная производная по $b$:
\[
\frac{\partial F}{\partial b} = -2 \sum_{i=1}^{n} (y_i - ax_i - b) = -2(S_y - aS_x - nb) = 0.
\]

Из этих уравнений формируем систему линейных уравнений:
\[
aS_{xx} + bS_x = S_{xy},
\]
\[
aS_x + nb = S_y.
\]

\subsubsection{Решение системы уравнений}
Решаем систему уравнений относительно $a$ и $b$:

1. Выражаем $b$ из второго уравнения:
\[
b = \frac{S_y - aS_x}{n}.
\]

2. Подставляем $b$ в первое уравнение:
\[
aS_{xx} + \left(\frac{S_y - aS_x}{n}\right)S_x = S_{xy}.
\]

3. Решаем уравнение относительно $a$:
\[
a = \frac{nS_{xy} - S_xS_y}{nS_{xx} - S_x^2}.
\]

4. Теперь, зная $a$, находим $b$:
\[
b = \frac{S_y - aS_x}{n}.
\]

Таким образом, мы получаем окончательные формулы для $a$ и $b$:
\[
a = \frac{nS_{xy} - S_xS_y}{nS_{xx} - S_x^2}, \quad b = \frac{S_y - aS_x}{n}.
\]

\subsection{Алгоритм метода наименьших квадратов}
Алгоритм линейной аппроксимации методом наименьших квадратов (МНК) включает следующие шаги:
\begin{enumerate}
  \item Вычислить предварительные суммы $S_x$, $S_y$, $S_{xy}$ и $S_{xx}$.
  \item Используя полученные суммы, рассчитать коэффициенты $a$ и $b$ по приведённым выше формулам.
  \item С помощью найденных коэффициентов $a$ и $b$ построить линейную функцию $y = ax + b$.
  \item Для оценки точности аппроксимации найти максимальное квадратичное отклонение между наблюдаемыми значениями $y_i$ и значениями, предсказанными моделью $ax_i + b$.
\end{enumerate}
