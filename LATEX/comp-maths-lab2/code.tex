\newpage
\section{Код численного метода}

\subsection{Реализация и анализ метода}
В этом разделе представлена программная реализация метода Гаусса с выбором главного элемента, написанная на языке программирования C++. Функция принимает на вход размер матрицы и саму матрицу, возвращает решение СЛАУ и невязки для оценки погрешности.

\subsection{Код функции}
\begin{lstlisting}
vector<double> solveByGauss(int n, vector<vector<double>> matrix) {
    vector<double> solution(n);
    vector<double> residuals(n);
    
    // Gauss elimination with pivoting
    for (int i = 0; i < n; ++i) {
        int maxRow = i;
        // Check columns for the row with the biggest value
        for (int j = i + 1; j < n; ++j) {
            if (abs(matrix[j][i]) > abs(matrix[maxRow][i])) {
                maxRow = j;
            }
        }
        
        // Check if matrix is singular
        if (abs(matrix[maxRow][i]) < 1e-10) {
            isSolutionExists = false;
            errorMessage = "The system has no roots of equations or has an infinite set of them.";
            return solution;
        }
        
        // swap rows
        swap(matrix[i], matrix[maxRow]);
        
        // Forward elimination
        for (int j = i + 1; j < n; ++j) {
            double factor = matrix[j][i] / matrix[i][i];
            for (int k = i; k <= n; ++k) {
                matrix[j][k] -= factor * matrix[i][k];
            }
        }
    }
    
    // Back substitution
    for (int i = n - 1; i >= 0; --i) {
        double sum = 0.0;
        for (int j = i + 1; j < n; ++j) {
            sum += matrix[i][j] * solution[j];
        }
        solution[i] = (matrix[i][n] - sum) / matrix[i][i];
    }
    
    // Compute residuals
    for (int i = 0; i < n; ++i) {
        double sum = 0.0;
        for (int j = 0; j < n; ++j) {
            sum += matrix[i][j] * solution[j];
        }
        residuals[i] = matrix[i][n] - sum;
    }
    
    // Add residuals in the end of the solution vector
    solution.insert(solution.end(), residuals.begin(), residuals.end());
    
    return solution;
}
\end{lstlisting}
