\section{Описание метода Гаусса с выбором главного элемента}

\subsection{Общие положения}
Метод Гаусса, названный в честь выдающегося математика Карла Фридриха Гаусса, остается одним из наиболее фундаментальных и широко применяемых алгоритмов для решения систем линейных алгебраических уравнений (СЛАУ). Его эффективность и гибкость лежат в основе современных численных методов в линейной алгебре и компьютерных науках.

\subsection{Усовершенствование классического метода Гаусса}
Вариант метода Гаусса с выбором главного элемента является эволюционным шагом в развитии оригинального алгоритма. Он интегрирует стратегию выбора максимального по модулю элемента в текущем столбце, что приводит к уменьшению ошибок округления и повышению общей численной стабильности. Из любопытства я решил реализовать этот вариант метода Гаусса, потому что применение такого подхода расширит кругозор и даст практический опыт в актуальных задачах при решении систем с плохо обусловленными или разреженными матрицами, где стандартный метод Гаусса может приводить к значительным вычислительным ошибкам.

\subsection{Теоретическая база}
Рассмотрим СЛАУ вида \(Ax = b\), где \(A\) -- квадратная матрица коэффициентов размером \(n \times n\), \(x\) -- вектор неизвестных, \(b\) -- вектор свободных членов.

\subsection{Выбор главного элемента}
На каждом шаге алгоритма Гаусса для текущего столбца выбирается элемент с наибольшим по модулю значением (главный элемент). Это обеспечивает уменьшение ошибок, вызванных округлением.

\subsection{Прямой ход}
\begin{itemize}
  \item Приведение матрицы \(A\) к верхнетреугольной форме.
  \item Выбор главного элемента в столбце.
  \item Обнуление элементов ниже главного элемента с помощью элементарных преобразований строк.
\end{itemize}

\subsection{Обратный ход}
\begin{itemize}
  \item Решение полученной верхнетреугольной системы путем подстановки известных значений и вычисления неизвестных начиная с последнего уравнения.
\end{itemize}

\subsection{Математическое изложение алгоритма}
Формализация шагов метода Гаусса предусматривает последовательное преобразование системы до достижения верхнетреугольной формы матрицы \( A \). Главный элемент \( a_{kk} \), обнаруженный на \( k \)-м шаге, определяет траекторию преобразований для элементов \( i > k \). Пересчет значений проводится согласно следующим формулам:
\begin{align}
    a_{ij}^{(k+1)} &= a_{ij}^{(k)} - \frac{a_{ik}^{(k)}}{a_{kk}^{(k)}} \cdot a_{kj}^{(k)}, & j &= k, \ldots, n, \\
    b_{i}^{(k+1)} &= b_{i}^{(k)} - \frac{a_{ik}^{(k)}}{a_{kk}^{(k)}} \cdot b_{k}^{(k)}.
\end{align}
Затем выполняется обратный ход, начиная с последнего уравнения системы, для вычисления значений вектора неизвестных \( \vec{x} \):
\begin{align}
    x_i &= \frac{1}{a_{ii}} \left( b_i - \sum_{j=i+1}^{n} a_{ij} x_j \right), & i &= n, n-1, \ldots, 1.
\end{align}

\subsection{Оценка точности решения}
Невязка системы \( \vec{r} \), вычисленная как разность между вектором свободных членов \( \vec{b} \) и произведением матрицы \( A \) на вектор найденных неизвестных \( \vec{x} \), служит мерой точности полученного решения:
\begin{equation}
    \vec{r} = \vec{b} - A\vec{x}.
\end{equation}
Малые значения невязки указывают на высокую точность решения.

\subsection{Завершающие положения}
Метод Гаусса с выбором главного элемента является устойчивым и эффективным  инструментом для решения СЛАУ. В следующих главах я приведу пример его практического применения с программной реализацией и анализами результатов.

\newpage