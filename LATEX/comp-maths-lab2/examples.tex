\section{Примеры использования кода}

В данном разделе представлены примеры работы программы на различных входных данных, включая граничные случаи и исключительные ситуации.

\subsection{Пример 1: Обычный случай}

Рассмотрим систему уравнений:
\[
\begin{cases} 
3x_1 + 2x_2 = 5 \\ 
4x_1 - x_2 = 3
\end{cases}
\]

После применения метода Гаусса с выбором главного элемента, получаем решение: $x_1 = 1, x_2 = 1$.

Для нашего первого примера расчет невязок выглядит следующим образом:

\[
r_1 = |3 \cdot 1 + 2 \cdot 1 - 5| = 0
\]
\[
r_2 = |4 \cdot 1 - 1 \cdot 1 - 3| = 0
\]

Невязки для обоих уравнений равны нулю, что свидетельствует о том, что найденное решение точно удовлетворяет исходной системе уравнений. Это подтверждает корректность применения метода Гаусса с выбором главного элемента и эффективность данного метода для решения СЛАУ.

\subsection{Пример 2: Система с бесконечным числом решений}

\[
\begin{cases} 
x_1 + 2x_2 = 3 \\ 
2x_1 + 4x_2 = 6
\end{cases}
\]

Эта система уравнений представляет собой два уравнения одной и той же прямой. Программная реализация вернет пустой вектор solution и установит isSolutionExists в значение false, указывая на то, что система имеет бесконечное множество решений.

Код корректно обрабатывает случай бесконечного числа решений, рассматривая матрицу как вырожденную, когда ведущий элемент близок к нулю. 

\subsection{Пример 3: Система без решений}

\[
\begin{cases} 
x_1 + 2x_2 = 3 \\ 
2x_1 + 4x_2 = 7
\end{cases}
\]

Аналогично второму примеру, код обнаружит, что матрица является вырожденной, так как ведущий элемент во второй строке будет близок к нулю после шага исключения. Поскольку ведущий элемент считается нулевым, код установит isSolutionExists в false, присвоит сообщение об ошибке в errorMessage и вернет пустой вектор solution.

\subsection{Пример 5: Сложная система с множеством переменных}

Рассмотрим следующую систему линейных уравнений с 5 переменными:

\[
\begin{cases}
2x_1 - 3x_2 + 5x_3 + x_4 - 2x_5 = 10 \\
4x_1 + 2x_2 - 3x_3 + 2x_4 + x_5 = 5 \\
-x_1 + 4x_2 + 2x_3 - 3x_4 + 5x_5 = -2 \\
3x_1 - 2x_2 + 6x_3 + 4x_4 - x_5 = 8 \\
5x_1 + 3x_2 - 2x_3 + x_4 + 4x_5 = 12
\end{cases}
\]

Решение, полученное с помощью функции \texttt{solveByGauss}:

\begin{verbatim}
x1 = 1.0000000000000009
x2 = 2.0000000000000027
x3 = 1.0000000000000009
x4 = 0.9999999999999998
x5 = -1.0000000000000009

Невязки:
r1 = -8.881784197001252e-16
r2 = 8.881784197001252e-16
r3 = -4.440892098500626e-16
r4 = 8.881784197001252e-16
r5 = -8.881784197001252e-16
\end{verbatim}

Результаты показывают, что функция \texttt{solveByGauss} успешно нашла решение системы уравнений. Значения невязок близки к нулю (порядка $10^{-16}$), что свидетельствует о высокой точности полученного решения. Небольшие отклонения от точного нуля обусловлены ограниченной точностью представления чисел с плавающей запятой в компьютере.

\section{Заключение}

В данной работе был реализован метод Гаусса с выбором главного элемента для решения систем линейных уравнений. Метод был протестирован на различных примерах, включая системы с единственным решением, с бесконечным множеством решений и без решений. Результаты показали, что реализованный алгоритм успешно справляется с поставленной задачей и находит точные решения с небольшими невязками, обусловленными ограниченной точностью представления чисел с плавающей запятой.

Временная сложность реализованного метода Гаусса с выбором главного элемента составляет $O(n^3)$, где $n$ - размерность системы уравнений. Это связано с необходимостью выполнения $n$ итераций, на каждой из которых производится поиск главного элемента и исключение переменных, что требует $O(n^2)$ операций. Таким образом, метод является эффективным для систем небольшой и средней размерности, но может быть недостаточно быстрым для больших систем.

По сравнению с другими методами решения систем линейных уравнений, метод Гаусса с выбором главного элемента имеет ряд преимуществ. В отличие от базового метода Гаусса без выбора главного элемента, данный метод является более устойчивым к погрешностям и позволяет избежать ситуаций, когда главный элемент близок к нулю, что может привести к неточным результатам. Метод Холецкого применим только для симметричных положительно определенных матриц, в то время как метод Гаусса с выбором главного элемента может работать с любыми невырожденными матрицами. Методы простой итерации и Гаусса-Зейделя могут сходиться медленно для некоторых типов систем, в то время как метод Гаусса с выбором главного элемента гарантированно находит точное решение за конечное число шагов.

В целом, метод Гаусса с выбором главного элемента является надежным и эффективным инструментом для решения систем линейных уравнений. Его реализация может быть полезна в различных областях, таких как численные методы, оптимизация и моделирование физических процессов. Дальнейшие улучшения могут включать в себя оптимизацию алгоритма для больших разреженных систем и распараллеливание вычислений для повышения производительности.

