\section{Примеры работы программы}

\subsection{Пример 1: Интегрирование линейной функции}

\textbf{Описание}: Вычисление интеграла линейной функции \(f(x) = 2x\) на интервале \([1, 3]\).

\textbf{Ввод}:
\begin{itemize}
    \item Функция: \(f(x) = 2x\)
    \item Нижний предел: 1
    \item Верхний предел: 3
    \item Точность: \(\epsilon = 1 \times 10^{-6}\)
\end{itemize}

\textbf{Вывод}: Результат интегрирования \(\approx 8.0\)

\textbf{Теоретическое значение}: \(\int_1^3 2x \, dx = 8\)

\textbf{Абсолютная ошибка}: \(|8 - 8.0| = 0\)

\textbf{Анализ численной ошибки}: Поскольку абсолютная ошибка равна 0, метод трапеций точно вычислил интеграл для линейной функции на данном интервале.

\subsection{Пример 2: Интегрирование функции с разрывом}

\textbf{Описание}: Вычисление интеграла функции \(f(x) = \frac{1}{x}\) на интервале \([0.1, 1]\).

\textbf{Ввод}:
\begin{itemize}
    \item Функция: \(f(x) = \frac{1}{x}\)
    \item Нижний предел: 0.1
    \item Верхний предел: 1
    \item Точность: \(\epsilon = 1 \times 10^{-6}\)
\end{itemize}

\textbf{Вывод}: Результат интегрирования \(\approx 2.302\)

\textbf{Теоретическое значение}: \(\int_{0.1}^1 \frac{1}{x} \, dx = \ln(10) \approx 2.302585\)

\textbf{Абсолютная ошибка}: \(|\ln(10) - 2.302| \approx 0.000585\)

\textbf{Анализ численной ошибки}: Малая абсолютная ошибка указывает на высокую точность метода трапеций для этого примера, несмотря на сильную кривизну функции.

\subsection{Пример 3: Интегрирование периодической функции}

\textbf{Описание}: Вычисление интеграла периодической функции \(f(x) = \sin(x)\) на интервале \([0, 2\pi]\).

\textbf{Ввод}:
\begin{itemize}
    \item Функция: \(f(x) = \sin(x)\)
    \item Нижний предел: 0
    \item Верхний предел: \(2\pi\)
    \item Точность: \(\epsilon = 1 \times 10^{-6}\)
\end{itemize}

\textbf{Вывод}: Результат интегрирования \(\approx 0.0\)

\textbf{Теоретическое значение}: \(\int_0^{2\pi} \sin(x) \, dx = 0\)

\textbf{Абсолютная ошибка}: \(|0 - 0.0| = 0\)

\textbf{Анализ численной ошибки}: Результат подтверждает высокую точность метода трапеций для функций с симметричными периодами.

\subsection{Пример 4: Интегрирование полиномиальной функции с высокой степенью}

\textbf{Описание}: Вычисление интеграла функции \(f(x) = x^4\) на интервале \([0, 1]\).

\textbf{Ввод}:
\begin{itemize}
    \item Функция: \(f(x) = x^4\)
    \item Нижний предел: 0
    \item Верхний предел: 1
    \item Точность: \(\epsilon = 1 \times 10^{-4}\)
\end{itemize}

\textbf{Вывод}: Результат интегрирования \(\approx 0.2\)

\textbf{Теоретическое значение}: \(\int_0^1 x^4 \, dx = \frac{1}{5} = 0.2\)

\textbf{Абсолютная ошибка}: \(|0.2 - 0.2| = 0\)

\textbf{Анализ численной ошибки}: Точность метода трапеций подтверждена для полиномиальной функции, демонстрируя его эффективность для функций с монотонным поведением на интервале.

\subsection{Пример 5: Интегрирование функции с разрывами}

\textbf{Описание}: Вычисление интеграла функции \(f(x) = \frac{1}{\sqrt{x}}\) на интервале \([0.01, 1]\), избегая неопределенности в начальной точке.

\textbf{Ввод}:
\begin{itemize}
    \item Функция: \(f(x) = \frac{1}{\sqrt{x}}\)
    \item Нижний предел: 0.01
    \item Верхний предел: 1
    \item Точность: \(\epsilon = 1 \times 10^{-6}\)
\end{itemize}

\textbf{Вывод}: Результат интегрирования \(\approx 4.0\)

\textbf{Теоретическое значение}: \(\int_{0.01}^1 \frac{1}{\sqrt{x}} \, dx = 4\)

\textbf{Абсолютная ошибка}: \(|4 - 4.0| = 0\)

\textbf{Анализ численной ошибки}: Отсутствие ошибки подчеркивает адаптивность метода трапеций при корректировке параметров интегрирования для функций с особенностями.

