\newpage
\section{Заключение}

В ходе лабораторной работы был применён метод трапеций для численного интегрирования различных функций. Асимптотическая сложность метода трапеций составляет \textbf{$O(n)$}, где $n$ — количество разбиений интервала интегрирования. Это означает, что время выполнения операции увеличивается \textbf{линейно} с увеличением числа трапеций.

Как \textit{метод прямоугольников}, так и \textit{метод трапеций} хорошо подходят для функций с малой кривизной на интервале интегрирования, как было показано на примере линейной функции, где метод показал высокую точность. В отличие от метода трапеций, метод прямоугольников хоть и имеет такую же временную сложность $O(n)$, но менее точен из-за его принципа использования только одного значения функции (обычно в середине интервала, поэтому он также называется "метод средних прямоугольников") для аппроксимации площади под кривой. Для функций с высокой кривизной или разрывами, как $\frac{1}{x}$ или $\frac{1}{\sqrt{x}}$, оба этих метода могут требовать значительно более мелкого разбиения для достижения адекватной точности, что увеличивает вычислительные затраты.

\textit{Метод Симпсона}, в отличие от предыдущих, имеет асимптотическую сложность \(O(n^4)\) для гладких функций, что обеспечивает значительно более высокую точность на функциях с высокой кривизной или переменными значениями, что были указаны выше. Однако, это также делает метод Симпсона более сложным и ресурсоемким в реализации.

\textbf{Вывод}:
Метод трапеций представляет собой хороший компромисс между сложностью и точностью, делая его подходящим для широкого спектра задач, где требуется умеренная точность и простота реализации. Для задач, требующих высокую точность на сложных функциях, метод Симпсона будет предпочтительнее. В то время как метод прямоугольников может быть полезен для быстрых оценочных вычислений, когда точность не является критичной.


