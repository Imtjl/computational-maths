\section{Описание метода}

Метод трапеций — это один из численных методов приближённого вычисления определённых интегралов. Основная идея метода заключается в разбиении интервала интегрирования \([a, b]\) на \(n\) равных частей с шагом \(h=\frac{b-a}{n}\). Подынтегральная функция \(f(x)\) аппроксимируется на каждом подинтервале прямой, соединяющей узловые точки, что приводит к замене графика функции на ломаную линию, проходящую через все узловые точки.

Значение функции вычисляется в узловых точках:
\[
y_i = f(x_i), \quad x_i = a + i \cdot h, \quad i = 0, 1, \ldots, n
\]

Отсюда сумма площадей \(S_i\) частичных трапеций, ограниченных ломаной линией и осью \(x\):
\[
S_i = \frac{1}{2}h(y_i + y_{i+1}), \quad i = 0, 1, \ldots, n-1
\]

Тогда приближенное значение интеграла равно:
\[
I \approx \sum \limits_{i=1}^{n-1}S_i = \frac{h}{2} \sum \limits_{i=1}^{n}(y_i + y_{i+1}) = \frac{h}{2}(y_0 + y_n + 2\sum \limits_{i=1}^{n-1}y_i).
\]

\textbf{Сходимость и применимость:} Метод трапеций эффективен для функций, которые являются непрерывными и достаточно гладкими на заданном интервале. Он имеет первый порядок сходимости, что означает линейную зависимость ошибки от ширины интервала разбиения. Ошибка метода составляет \(O(h^2)\), если функция дважды дифференцируема на интервале интегрирования.

