\newpage
\section{Код численного метода интегрирования}

\subsection{Реализация и анализ метода}
Метод численного интегрирования реализован на C++ и используется для вычисления определённого интеграла функции. В основе метода лежит подход трапеций, который итеративно уточняет значение интеграла до достижения заданной точности. Ключевым моментом является разбиение интервала интегрирования на увеличивающееся количество трапеций и вычисление суммы площадей этих трапеций.

\subsection{Описание переменных и констант}
В функции используются следующие переменные и параметры для контроля точности вычислений и управления процессом интегрирования:

\begin{itemize}
    \item \textbf{lower\_bound, upper\_bound} — Нижний и верхний пределы интегрирования соответственно.
    \item \textbf{f} — Идентификатор функции, которая будет интегрироваться.
    \item \textbf{epsilon} — Заданная точность вычисления интеграла. Итерационный процесс продолжается до тех пор, пока разность между двумя последовательными приближениями интеграла не станет меньше этого значения.
\end{itemize}

\subsection{Код функции}
\begin{lstlisting}
double calculate_integral(double lower_bound, double upper_bound, int f, double epsilon) {
    fn_t& function = get_function(f);

    // Check for discontinuity or undefined region
    if ((f == 1 && (lower_bound <= 0 || upper_bound <= 0)) || (f == 5 && (lower_bound <= 0 || upper_bound <= 0))) {
        error_message = "Integrated function has discontinuity or does not defined in current interval";
        has_discontinuity = true;
        return 0.0;
    }

    int num_trapezoids = 1;
    double integral = 0.0;
    double prev_integral = 0.0;

    do {
        prev_integral = integral;
        integral = 0.0;
        double trapezoid_width = (upper_bound - lower_bound) / num_trapezoids;

        for (int i = 0; i < num_trapezoids; i++) {
            double x1 = lower_bound + i * trapezoid_width;
            double x2 = lower_bound + (i + 1) * trapezoid_width;
            integral += 0.5 * (function(x1) + function(x2)) * trapezoid_width;
        }

        num_trapezoids *= 2;
    } while (std::abs(integral - prev_integral) > epsilon);

    return integral;
}
\end{lstlisting}