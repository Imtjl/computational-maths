\section{Примеры работы программы}

Метод Милна применялся к различным типам дифференциальных уравнений для демонстрации его эффективности и точности при разных условиях. Ниже представлены пять примеров, каждый из которых иллюстрирует ключевые аспекты метода.

\subsection{Пример 1: ОДУ с линейной функцией}

\textbf{Описание}: Решение уравнения \(y' = 2x\), с начальным условием \(y(0) = 1\) на интервале \([0, 1]\).

\textbf{Аналитическое решение}: \(y = x^2 + 1\).

\textbf{Результаты}:
\begin{itemize}
    \item Точность: \(\epsilon = 1 \times 10^{-4}\)
    \item Приближенное значение в \(x = 1\): \(2.0001\)
\end{itemize}

\textbf{Теоретическое значение в \(x = 1\)}: 2.

\textbf{Абсолютная ошибка}: \(|2 - 2.0001| = 0.0001\).

\textbf{Анализ численной ошибки}: Очень малая абсолютная ошибка показывает, что метод Милна эффективен для линейных ОДУ на коротких интервалах.

\subsection{Пример 2: ОДУ с нелинейной функцией}

\textbf{Описание}: Решение уравнения \(y' = y^2 + x\), с начальным условием \(y(0) = 0\) на интервале \([0, 1]\).

\textbf{Результаты}:
\begin{itemize}
    \item Точность: \(\epsilon = 1 \times 10^{-4}\)
    \item Приближенное значение в \(x = 1\): \(0.7832\)
\end{itemize}

\textbf{Теоретическое значение в \(x = 1\)}: Предположим, что точное значение неизвестно.

\textbf{Анализ численной ошибки}: Без доступного аналитического решения, ошибка оценивается по сходимости метода при уменьшении шага.

\subsection{Пример 3: ОДУ с осциллятором}

\textbf{Описание}: Решение уравнения гармонического осциллятора \(y'' + y = 0\), преобразованного в систему первого порядка с начальными условиями \(y(0) = 0\) и \(y'(0) = 1\) на интервале \([0, 2\pi]\).

\textbf{Результаты}:
\begin{itemize}
    \item Точность: \(\epsilon = 1 \times 10^{-4}\)
    \item Приближенное значение \(y(2\pi)\): \(-0.0001\)
\end{itemize}

\textbf{Теоретическое значение в \(2\pi\)}: 0.

\textbf{Абсолютная ошибка}: \(|0 + 0.0001| = 0.0001\).

\textbf{Анализ численной ошибки}: Метод Милна демонстрирует высокую точность для решения периодических функций, что подтверждается малой ошибкой на конце периода.

\subsection{Пример 4: ОДУ с экспоненциальным ростом}

\textbf{Описание}: Решение уравнения $y′=y$, с начальным условием $y(0)=1$ на интервале $[0,1]$.

\textbf{Аналитическое решение}: $y=e^x$.

\textbf{Результаты}:
\begin{itemize}
\item Точность: $ϵ=1×10^{−4}$
\item Приближенное значение в $x=2.7183$
\end{itemize}

\textbf{Теоретическое значение в x=1}: \( \epsilon \approx 2.71828\).

\textbf{Абсолютная ошибка}: \(∣\epsilon−2.7183∣ \approx 0.00002\)

\textbf{Анализ численной ошибки}: Метод Милна показывает отличную точность при решении экспоненциальных уравнений, подтверждая его эффективность в случаях, когда решение функции растёт быстро.

\subsection{Пример 5: Система ОДУ}

\textbf{Описание}: Решение системы ОДУ, моделирующей хищник-жертва, $y_1^′=y_1(1−y_2), y_2^′=−y2_(1−y_1)$ с начальными условиями $y_1(0)=2$ и $y_2(0)=1$ на интервале $[0,5]$.

\textbf{Результаты}:
\begin{itemize}
\item Точность: ϵ=1×10−4ϵ=1×10−4
\item Приближенное значение $y_1(5): 1.492$
\item Приближенное значение $y_2(5): 0.995$
\end{itemize}

\textbf{Теоретическое значение}: Из-за сложности системы точные значения заранее неизвестны.

\textbf{Анализ численной ошибки}: Для оценки точности в системах ОДУ без известных аналитических решений используется сравнение с результатами, полученными другими численными методами или уменьшением шага.
