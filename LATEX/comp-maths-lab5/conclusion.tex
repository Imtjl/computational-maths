\section{Заключение}

В ходе лабораторной работы был исследован метод Милна для численного решения обыкновенных дифференциальных уравнений. Метод Милна принадлежит к классу многошаговых методов и использует технику предиктор-корректор для повышения точности решения при заданном шаге сетки.

\subsection{Вычислительная сложность}

Вычислительная сложность метода Милна зависит от количества требуемых операций на каждом шаге. Так как метод использует предыдущие четыре точки для вычисления следующего значения, его сложность можно оценить как \(O(n)\), где \(n\) — количество шагов интегрирования. Однако, из-за необходимости корректировки, на каждом шаге выполняется несколько дополнительных вычислений, что делает его менее эффективным по сравнению с некоторыми одношаговыми методами.

\subsection{Сравнение с другими методами}

Метод Милна, как и другие многошаговые методы, обычно предоставляет более высокую точность по сравнению с одношаговыми методами, такими как метод Эйлера. Метод Эйлера, хотя и является простым в реализации и имеет низкую вычислительную сложность \(O(n)\), страдает от низкой точности и большой численной ошибки при использовании на длительных интервалах или при решении уравнений с резкими изменениями поведения.

Методы Адамса, аналогичные по своей природе методу Милна, предлагают различные вариации многошаговых подходов. Методы Адамса-Башфорта, например, используют взвешенное среднее предыдущих значений функции для предсказания новых точек, что обеспечивает хорошее сочетание стабильности и точности.

Метод Милна, будучи предиктор-корректорным методом, предлагает дополнительное улучшение предсказания за счёт последующей коррекции, что делает его более точным в некоторых сценариях, особенно при решении задач с непрерывными, но быстро меняющимися функциями.


