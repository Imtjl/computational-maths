\newpage
\section{Программная реализация метода Милна}

\subsection{Реализация и анализ метода}

Метод Милна реализован в виде последовательности шагов, начиная с инициализации начальных точек с помощью метода Рунге-Кутта четвертого порядка, за которым следуют итерации предиктора и корректора Милна. Это обеспечивает точное и эффективное решение дифференциального уравнения.

\subsection{Описание переменных и констант}

\begin{itemize}
    \item \textbf{fn\_t\& f} — Функциональный тип, представляющий правую часть дифференциального уравнения \( y' = f(x, y) \), где \( x \) и \( y \) — независимая и зависимая переменные соответственно.
    \item \textbf{double x, double y} — Текущие значения независимой переменной \( x \) и зависимой переменной \( y \), используемые в методах Рунге-Кутта и Милна.
    \item \textbf{double h} — Шаг сетки для численного решения, который может адаптироваться в процессе выполнения программы для улучшения точности.
    \item \textbf{double epsilon} — Заданная точность решения, используемая для контроля точности корректора Милна.
    \item \textbf{double a, double y\_a, double b} — Начальные и конечные точки интервала интегрирования \( a \) и \( b \), а также начальное значение функции \( y(a) = y_a \).
\end{itemize}

\subsection{Код функции}

Основные функции в программной реализации:

\begin{itemize}
    \item \textbf{rungeKutta4} — Функция для вычисления начальных точек методом Рунге-Кутта четвертого порядка.
    \item \textbf{generateStartingPoints} — Функция для генерации начального набора значений, необходимых для начала работы многошагового метода Милна.
    \item \textbf{milnePredictor} и \textbf{milneCorrector} — Функции предиктора и корректора Милна соответственно, обеспечивающие численное приближение решения ОДУ.
    \item \textbf{solveByMilne} — Основная функция для решения ОДУ на заданном интервале с использованием метода Милна.
\end{itemize}
\subsection{Код функции}
\begin{lstlisting}
double rungeKutta4(fn_t& f, double x, double y, double h) {
    double k1 = h * f(x, y);
    double k2 = h * f(x + 0.5 * h, y + 0.5 * k1);
    double k3 = h * f(x + 0.5 * h, y + 0.5 * k2);
    double k4 = h * f(x + h, y + k3);
    return y + (k1 + 2 * k2 + 2 * k3 + k4) / 6;
}

std::vector<double> generateStartingPoints(fn_t& f, double a, double y_a, double h) {
    std::vector<double> points;
    points.push_back(y_a);
    double x = a;
    double y = y_a;
    for (int i = 0; i < 3; ++i) {
        y = rungeKutta4(f, x, y, h);
        points.push_back(y);
        x += h;
    }
    return points;
}

double milnePredictor(double y1, double y2, double y3, double y4, double h, fn_t& f, double x) {
    return y4 + 4.0 * h * (2 * f(x, y3) - f(x - h, y2) + 2 * f(x - 2 * h, y1)) / 3.0;
}

double milneCorrector(double y2, double y3, double y4, double y_p, double h, fn_t& f, double x) {
    return y3 + h * (f(x, y4) + 4 * f(x - h, y3) + f(x - 2 * h, y2)) / 3.0;
}

double solveByMilne(int f, double epsilon, double a, double y_a, double b) {
    fn_t& func = get_function(f);
    double h = (b - a) / 100;  // Начальный шаг
    
    std::vector<double> startingPoints = generateStartingPoints(func, a, y_a, h);
    double x = a + 3 * h;
    double y = startingPoints.back();
    
    while (x < b) {
        double y_pred = milnePredictor(startingPoints[0], startingPoints[1], startingPoints[2], startingPoints[3], h, func, x + h);
        double y_corr = milneCorrector(startingPoints[1], startingPoints[2], startingPoints[3], y_pred, h, func, x + h);
        
        if (std::abs(y_corr - y_pred) < epsilon) {
            startingPoints.erase(startingPoints.begin());
            startingPoints.push_back(y_corr);
            x += h;
            y = y_corr;
        } else {
            h /= 2;  // Уменьшаем шаг
            startingPoints = generateStartingPoints(func, x - 3 * h, startingPoints[3], h);
        }
    }
    
    return y;
}
\end{lstlisting}