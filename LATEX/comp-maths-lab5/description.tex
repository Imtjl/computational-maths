\section{Описание метода}

Метод Милна — это численный метод решения задачи Коши для обыкновенных дифференциальных уравнений (ОДУ). Этот метод относится к классу многошаговых методов предиктор-корректор и используется для увеличения точности и стабильности численного решения.

\subsection{Основные принципы}

Задача Коши для ОДУ формулируется следующим образом:
\[
y'(x) = f(x, y(x)), \quad y(a) = y_a,
\]
где \(y(x)\) — искомая функция, \(f(x, y)\) — заданная функция, производная которой определяется в уравнении, \(a\) и \(y_a\) — начальные условия.

\subsection{Метод Милна}

Метод Милна включает два основных шага: предиктор и корректор. Предиктор используется для приближенного вычисления значения функции, а корректор уточняет это значение.

\subsubsection{Предиктор}
На этапе предиктора вычисление нового значения \(y_{n+1}\) осуществляется по формуле:
\[
y_{n+1}^{pred} = y_{n-3} + \frac{4h}{3} \left( 2f(x_{n-1}, y_{n-1}) - f(x_{n-2}, y_{n-2}) + 2f(x_{n-3}, y_{n-3}) \right),
\]
где \(h\) — шаг сетки, \(x_n = a + nh\).

\subsubsection{Корректор}
Корректор уточняет значение, полученное предиктором, используя следующую формулу:
\[
y_{n+1} = y_{n-1} + \frac{h}{3} \left( f(x_{n+1}, y_{n+1}^{pred}) + 4f(x_n, y_n) + f(x_{n-1}, y_{n-1}) \right).
\]

\textbf{Инициализация метода}: Для начала работы метода Милна необходимо иметь начальные значения \(y_0, y_1, y_2, y_3\), которые обычно вычисляются с помощью метода Рунге-Кутта четвертого порядка.

\textbf{Применимость и сходимость}: Метод Милна хорошо работает на гладких функциях и обеспечивает высокую точность. Однако, как и другие многошаговые методы, он может быть чувствителен к начальным значениям и требует аккуратного выбора шага интегрирования для обеспечения стабильности и точности решения.

\textbf{Сравнение с другими методами}: Метод Милна, благодаря использованию информации о предыдущих точках, зачастую более точен, чем одношаговые методы, такие как метод Эйлера или классический метод Рунге-Кутта. В то же время, он требует больше вычислений на шаг по сравнению с методом Рунге-Кутта того же порядка точности из-за дополнительного корректирующего шага.
