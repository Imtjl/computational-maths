\newpage
\section{Заключение}

В данной работе был детально рассмотрен и реализован метод Ньютона для решения систем нелинейных алгебраических уравнений (СНАУ). В отличие от систем линейных уравнений (СЛАУ), решение СНАУ требует итеративного подхода из-за нелинейности функций, что существенно усложняет задачу. Метод Ньютона позволяет находить корни СНАУ, используя линеаризацию системы уравнений вокруг текущего приближения и последующее решение линейной системы методом Гаусса для коррекции этого приближения.

Асимптотическая сложность алгоритма Ньютона преимущественно зависит от сложности вычисления Якобиана и решения СЛАУ на каждой итерации. Сложность вычисления Якобиана может достигать \(O(n^3)\), аналогично методу Гаусса для решения СЛАУ, где \(n\) — количество переменных в системе. Таким образом, общая вычислительная сложность метода Ньютона также оценивается как \(O(n^3)\), учитывая, что количество итераций до сходимости обычно мало.

Сравнивая метод Ньютона с другими методами решения СНАУ, такими как метод простой итерации или метод секущих, можно отметить его более высокую скорость сходимости при условии хорошего начального приближения. В то же время, метод требует вычисления Якобиана на каждом шаге, что может быть вычислительно затратно для больших систем. Важным преимуществом метода Ньютона является его квадратичная сходимость, что обеспечивает быстрое приближение к точному решению после нескольких итераций, если начальное приближение выбрано достаточно близко к корню.

Метод Ньютона применяется к СНАУ, а не к СЛАУ, потому что он позволяет эффективно находить корни нелинейных систем, где линейные методы неприменимы из-за отсутствия явной формы решения. Это делает метод Ньютона ценным инструментом в различных областях науки и инженерии, где нелинейные системы встречаются повсеместно.

В заключение, метод Ньютона демонстрирует высокую эффективность и универсальность в решении СНАУ. Его способность быстро сходиться к точному решению, будучи правильно настроенным и применённым, делает его незаменимым инструментом для исследователей и инженеров. Несмотря на некоторые вычислительные трудности, связанные с большими системами, метод Ньютона остаётся золотым стандартом для реш
