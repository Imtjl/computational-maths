\section{Блок-схема алгоритма Ньютона для решения СНАУ}

\begin{tikzpicture}[node distance=2cm and 3cm, auto]
    
    % Nodes
    \node (start) [startstop] {Начало};
    \node (init) [prep, below=of start] {Инициализация параметров};
    \node (assign) [process, below=of init] {Выбор начального приближения \( x^{(0)} \)};
    \node (jacob) [process, below=of assign] {Вычисление Якобиана \( J(x^{(k)}) \)};
    \node (func) [process, below=of jacob] {Вычисление \( F(x^{(k)}) \)};
    \node (solve) [process, below=of func] {Решение СЛАУ методом Гаусса};
    \node (update) [process, below=of solve] {Обновление приближения \( x^{(k+1)} = x^{(k)} + \Delta x \)};
    \node (conv) [decision, below=of update, yshift=-1.5cm] {Сходится?};
    \node (end) [startstop, right=of conv, xshift=3cm] {Конец};
    
    % Arrows
    \draw [arrow] (start) -- (init);
    \draw [arrow] (init) -- (assign);
    \draw [arrow] (assign) -- (jacob);
    \draw [arrow] (jacob) -- (func);
    \draw [arrow] (func) -- (solve);
    \draw [arrow] (solve) -- (update);
    \draw [arrow] (update) -- (conv);
    \draw [arrow] (conv) -- node[anchor=south] {да} (end);
    \draw [arrow] (conv) -- node[near start, above] {нет} ++(-5.5,0) |- (assign);

\end{tikzpicture}
