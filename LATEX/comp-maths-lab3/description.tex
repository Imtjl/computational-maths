\section{Описание метода Ньютона}

Метод Ньютона, также известный как метод Ньютона-Рафсона, — это мощный итерационный численный метод для нахождения приближённых корней нелинейных уравнений и систем уравнений. Основная идея метода состоит в построении касательных к функции в предполагаемой точке корня и использовании точки пересечения касательной с осью абсцисс в качестве следующего приближения к корню.

\subsection{Метод Ньютона для одного уравнения}

Рассмотрим нелинейное уравнение вида \( f(x) = 0 \). Начнём с некоторого начального приближения \( x_0 \). Тогда следующее приближение \( x_{n+1} \) можно найти по формуле:
\[
	x_{n+1} = x_n - \frac{f(x_n)}{f'(x_n)},
\]
где \( f'(x_n) \) — производная функции \( f \) в точке \( x_n \).

\subsection{Метод Ньютона для системы уравнений}

Для системы нелинейных уравнений метод Ньютона использует аналогичный подход. Пусть дана система уравнений \( \mathbf{F}(\mathbf{x}) = \mathbf{0} \), где \( \mathbf{F}: \mathbb{R}^n \to \mathbb{R}^n \) и \( \mathbf{x} \) — вектор неизвестных. На каждой итерации \( k \) алгоритма выполняются следующие шаги:

\begin{enumerate}
	\item Вычисляется матрица Якоби \( J(\mathbf{x}^{(k)}) \), которая является матрицей всех первых частных производных вектор-функции \( \mathbf{F} \):
	      \[
		      J(\mathbf{x}^{(k)}) = \begin{bmatrix}
			      \frac{\partial F_1}{\partial x_1} & \cdots & \frac{\partial F_1}{\partial x_n} \\
			      \vdots                            & \ddots & \vdots                            \\
			      \frac{\partial F_n}{\partial x_1} & \cdots & \frac{\partial F_n}{\partial x_n}
		      \end{bmatrix}_{\mathbf{x} = \mathbf{x}^{(k)}}.
	      \]

	\item Вычисляется вектор функций \( \mathbf{F}(\mathbf{x}^{(k)}) \).

	\item Решается линейная система уравнений \( J(\mathbf{x}^{(k)}) \Delta \mathbf{x} = -\mathbf{F}(\mathbf{x}^{(k)}) \) для нахождения вектора поправок \( \Delta \mathbf{x} \).

	\item Обновляется вектор неизвестных: \( \mathbf{x}^{(k+1)} = \mathbf{x}^{(k)} + \Delta \mathbf{x} \).
\end{enumerate}

Итерации продолжаются до тех пор, пока не будет достигнута желаемая точность, т.е. пока \( \|\Delta \mathbf{x}\| < \varepsilon \) и \( \|\mathbf{F}(\mathbf{x}^{(k+1)})\| < \varepsilon \), где \( \varepsilon \) — заданная величина точности.

\subsection{Сходимость метода}

Метод Ньютона сходится квадратично в окрестности корня при условии, что начальное приближение достаточно близко к истинному решению и что \( J(\mathbf{x}^{(k)}) \) невырождена. Однако, метод может расходиться, если начальное приближение выбрано далеко от корня или функция \( \mathbf{F} \) имеет сложную структуру.

\subsection{Применение метода}

Метод Ньютона широко используется в инженерных и научных расчётах для решения различных нелинейных задач, включая оптимизацию, нахождение корней дифференциальных уравнений и многие другие области.
