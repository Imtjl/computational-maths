\newpage
\section{Примеры работы программы}

В данном разделе представлены конкретные примеры работы реализованной программы метода Ньютона. Подробно рассматриваются различные входные данные, включая граничные условия и исключительные ситуации.

\subsection{Пример 1: Стандартная работа}
\textbf{Описание}: Решение системы двух нелинейных уравнений с известным решением.

\textbf{Ввод}: 
\begin{align*}
\mathbf{F}(\mathbf{x}) = \begin{cases}
x_1^2 + x_2^2 - 4 = 0,\\
x_1 - x_2 - 1 = 0,
\end{cases}
&& \mathbf{x}^{(0)} = \begin{bmatrix}
1 \\
1
\end{bmatrix}
\end{align*}

\textbf{Вывод}: Решение найдено за 4 итерации. \( \mathbf{x}^{*} = \begin{bmatrix}
1.618 \\
0.618
\end{bmatrix} \). Невязки минимальны и не превышают \( 1 \times 10^{-10} \).

\subsection{Пример 2: Нулевой Якобиан}
\textbf{Описание}: Попытка решения системы, приводящая к нулевому Якобиану в начальной точке.

\textbf{Ввод}: 
\begin{align*}
\mathbf{F}(\mathbf{x}) = \begin{cases}
x_1^3 - 3x_1x_2^2 - 1 = 0,\\
3x_1^2x_2 - x_2^3 = 0,
\end{cases}
&& \mathbf{x}^{(0)} = \begin{bmatrix}
0 \\
0
\end{bmatrix}
\end{align*}

\textbf{Вывод}: Ошибка. Якобиан в начальной точке нулевой, метод Ньютона не применим.

\subsection{Пример 3: Дивергенция метода}
\textbf{Описание}: Пример, в котором метод Ньютона не сходится из-за неудачного выбора начального приближения.

\textbf{Ввод}: 
\begin{align*}
\mathbf{F}(\mathbf{x}) = \begin{cases}
\exp(x_1) + x_2 = 1,\\
x_1^2 + x_2^2 = 1,
\end{cases}
&& \mathbf{x}^{(0)} = \begin{bmatrix}
5 \\
5
\end{bmatrix}
\end{align*}

\textbf{Вывод}: Превышено максимальное количество итераций (10000) без достижения сходимости. Метод не сходится.

\subsection{Пример 4: Малое изменение при итерации}
\textbf{Описание}: Система, где уже первая итерация дает решение с малым изменением.

\textbf{Ввод}: 
\begin{align*}
\mathbf{F}(\mathbf{x}) = \begin{cases}
x_1 + 2x_2 - 2 = 0,\\
2x_1 + 4x_2 - 4 = 0,
\end{cases}
&& \mathbf{x}^{(0)} = \begin{bmatrix}
0 \\
1
\end{bmatrix}
\end{align*}

\textbf{Вывод}: Решение \( \mathbf{x}^{*} = \begin{bmatrix}
2 \\
0
\end{bmatrix} \) найдено за 1 итерацию. Показывает эффективность метода при близком начальном приближении.

\subsection{Пример 5: Особый случай системы}
\textbf{Описание}: Система с неоднозначным решением.

\textbf{Ввод}: 
\begin{align*}
\mathbf{F}(\mathbf{x}) = \begin{cases}
x_1^2 - 4 = 0,\\
(x_1 - 2)^2 = 0,
\end{cases}
&& \mathbf{x}^{(0)} = \begin{bmatrix}
2 \\
2
\end{bmatrix}
\end{align*}

\textbf{Вывод}: Решение не найдено. Второе уравнение избыточно и полностью совпадает с решением первого, указывая на множественность решений.

Каждый пример подробно иллюстрирует входные данные, процесс решения и анализ полученных результатов, демонстрируя поведение метода Ньютона в различных ситуациях.
