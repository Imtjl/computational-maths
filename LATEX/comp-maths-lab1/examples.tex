\section{Примеры использования кода}

\subsection{Пример 1: Линейная зависимость}
Входные данные:
\begin{lstlisting}
xVals = {1, 2, 3, 4, 5}
yVals = {2, 4, 6, 8, 10}
\end{lstlisting}
Результат: максимальное квадратичное отклонение = 0.0.

\subsection{Пример 2: Нелинейная зависимость}
Входные данные:
\begin{lstlisting}
xVals = {1, 2, 3, 4, 5}
yVals = {1, 4, 9, 16, 25}
\end{lstlisting}
Результат: максимальное квадратичное отклонение = 11.36.

\subsection{Пример 3: Случайные данные}
Входные данные:
\begin{lstlisting}
xVals = {2.5, 3.7, 1.2, 4.9, 6.1}
yVals = {7.3, 10.2, 3.8, 14.5, 18.9}
\end{lstlisting}
Результат: максимальное квадратичное отклонение = 0.2471.

\subsection{Пример 4: Граничный случай (мало точек)}
Входные данные:
\begin{lstlisting}
xVals = {1}
yVals = {2}
\end{lstlisting}
Результат: максимальное квадратичное отклонение = 0.0 (недостаточно точек для аппроксимации).

\subsection{Пример 5: Исключительная ситуация (вертикальная прямая)}
Входные данные:
\begin{lstlisting}
xVals = {2, 2, 2, 2, 2}
yVals = {1, 3, 5, 7, 9}
\end{lstlisting}
Результат: максимальное квадратичное отклонение = 0.0 (точки лежат на вертикальной прямой).